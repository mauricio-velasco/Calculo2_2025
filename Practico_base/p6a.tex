\documentclass[11pt]{article}

\usepackage[spanish]{babel}

\usepackage{amsfonts, amsthm, amssymb, amsmath,multicol}

\usepackage{graphics,graphicx}

\headsep17mm
\topmargin-1cm
\hoffset -1.5cm \voffset -1cm \textwidth 17cm \oddsidemargin
1.5cm \evensidemargin 1.5cm \textheight 22.5cm

\newcommand{\coef}[2]{\left( \begin{array}{c} #1 \\ #2 \end{array}\right)}

\newcounter{cuent}
%\newcommand{\proba}[1]{\stepcounter{cuent}{\alph{cuent})\quad}
%\displaystyle#1\qquad}
\newcommand{\proba}[1]{\stepcounter{cuent}{\alph{cuent})\ }
\displaystyle#1\hfill}
\newcommand{\cuento}{\setcounter{cuent}{0}}
\newcommand{\R}{\mathbb{R}}
\newcommand{\be}{\begin{enumerate}}
\newcommand{\ee}{\end{enumerate}}

\begin{document}

\noindent {\bf Universidad de la Rep\'{u}blica} \hfill {\bf C\'{a}lculo 2} \\
{\bf Facultad de Ciencias} 
{\bf Centro de Matem\'{a}tica} 

\vspace{1cm}

\begin{center}
{\bf Pr\'{a}ctico 6A [P6A]}
\end{center}

\vspace{0,1cm}

\begin{enumerate}



\item Para las siguientes funciones calcular los polinomios de Taylor de órdenes 1 y 2 en el punto indicado. 
a) $f(x,y)=\sen(x)\sen(y)$ en $(0,0)$ \ \ \  b) $f(x,y)=\sqrt{5x+2y}$ en $(1,1)$ \ \ \  c) $f(x,y)=e^{x^2+y(y+1)}$ en $(0,0)$ 

\item Para las siguientes funciones calcular los polinomios de Taylor de órden $n$ (arbitrario) en el punto indicado.

a) $f(x,y,z)=xyz$ en $(1,-1,0)$ \ \ \ b) $f(x,y)=e^{x+y}$ en
  $(0,0)$ y $(1,0)$.

\item Calcular los siguientes l\'imites:

a) $\underset{\left( x,y\right) \rightarrow \left( 0,0\right) }{\lim}\frac{xy-\sen(x)\sen(y)}{x^2+y^2}$\ \ 
b) $\underset{\left( x,y\right) \rightarrow \left( 0,0\right) }{\lim}\frac{e^{x^2+y(y+1)}-(1 + y)}{x^2+y^2}$.

  
  \item Si $\mathbf{r_1}$ y $\mathbf{r_2}$ son las distancias desde un
  punto $(x,y)$ de una elipse a sus focos, demostrar que la ecuaci\'on
  $\mathbf{r}_1+\mathbf{r}_2=$constante (que satisfacen esas
  distancias, y que es la definición de elipse) implica la relaci\'on
  $T\cdot\nabla(\mathbf{r}_1+\mathbf{r}_2)=0$, siendo $T$ el vector
  unitario tangente a la elipse en $(x,y)$. Interpretar geom\'etricamente este
  resultado, y con ello demostrar que la tangente forma \'angulos
  iguales con las rectas que unen $(x,y)$ a los focos. 
\end{enumerate}
\end{document}
