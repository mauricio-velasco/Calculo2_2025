\documentclass[11pt]{article}

\usepackage[spanish]{babel}

\usepackage{amsfonts, amsthm, amssymb, amsmath,multicol}

\usepackage{graphics,graphicx}

\headsep17mm
\topmargin-1cm
\hoffset -1.5cm \voffset -1cm \textwidth 17cm \oddsidemargin
1.5cm \evensidemargin 1.5cm \textheight 22.5cm

\newcommand{\coef}[2]{\left( \begin{array}{c} #1 \\ #2 \end{array}\right)}

\newcounter{cuent}
%\newcommand{\proba}[1]{\stepcounter{cuent}{\alph{cuent})\quad}
%\displaystyle#1\qquad}
\newcommand{\proba}[1]{\stepcounter{cuent}{\alph{cuent})\ }
\displaystyle#1\hfill}
\newcommand{\cuento}{\setcounter{cuent}{0}}
\newcommand{\R}{\mathbb{R}}
\newcommand{\RR}{\mathbb{R}}

\begin{document}

\noindent {\bf Universidad de la Rep\'{u}blica} \hfill {\bf C\'{a}lculo 2} \\
{\bf Facultad de Ciencias}  \\
{\bf Centro de Matem\'{a}tica} 

\vspace{1cm}

\begin{center}
{\bf Pr\'{a}ctico [P1] }
\end{center}

\vspace{0,1cm}

\begin{enumerate}

\item Representar gr\'aficamente los siguientes conjuntos de $\R^2$. Cuando sea posible, reconocer geom\'etricamente el conjunto.

\begin{enumerate}

\item $A=\{(x,y)\in\R^2:\ y=\sen(x),\ \frac{-\pi}{2}\leq x \leq \frac{\pi}{2}\}$.

\item $A=\{(x,y)\in\R^2:\ -1\leq x \leq y^2,\ 2\leq y < 3\}$.

\item $A=\{(x,y)\in\R^2:\ -\sqrt{1-x^2}\leq y \leq \sqrt{1-x^2},\ -1\leq x \leq 1\}$.

\item $A=\{(x,y)\in\R^2:\ 3x+2y=-1\}$.

\item $A=\{(x,y)\in\R^2:\ x-5y=-1,\ -2\leq x < 2\}$.

\item $A=\{(x,y)\in\R^2:\ x^2+y^2-2x-4y+5=3\}$.

\item $A=\{(x,y)\in\R^2:\ x=y^2-2y+1,\ -1\leq x \leq 1\}$.

\item $A=\{(x,y)\in\R^2:\ 5x^2+2y^2=2\}$.

\item $A=\{(x,y)\in\R^2:\ -3x^2+y^2=1\}$.

\item $A=\{(x,y)\in\R^2:\ x=5\cos(\theta),\ y=5\sen(\theta),\ \theta\in[0,\pi]\}$.


\end{enumerate}


\item Dados los vectores $u=(1,1),\ v=(2,1)$ de $\R^2$ hallar

$\proba{\parallel u + v \parallel} \proba{\text{dist}(u,v)}
\proba{\left<u,v\right>} \proba{\mbox{ el \'angulo entre } u \mbox{ y } v}$\\
$\proba{\mbox{ la recta } r \mbox{ paralela a }u\mbox{ por }(0,1)}  \proba{\mbox{ la recta } s \mbox{ perpendicular a }v\mbox{ por }(0,0)}$\\
$\proba{r\cap s}.$

\cuento

\vspace{0,4cm}

\item Muestre que en general dado un vector $v$ del plano $\R^2$, el conjunto $K_v$ de puntos cuya proyecci\'on sobre $v$ es cero es una recta que pasa por el origen.
Encuentre dicha recta para $v=(2,3)$. Muestre que dados dos vectores $v$ y $v'$, dichas rectas coinciden si y solo si $v$ y $v'$ son colineales. 

\item Represente gr\'aficamente los siguientes conjuntos del espacio $\R^3$. Cuando sea posible, reconocer geom\'etricamente el conjunto. 

\begin{enumerate}

\item $A=\{(x,y,z)\in\R^3:\ 3x+2y-z=1\}$.

\item $A=\{(x,y,z)\in\R^3:\ x+y-z=0,\ -2x-2y+2z=1\}$.

\item $A=\{(x,y,z)\in\R^3:\ x=t,\ y=2t,\ z=-t\mbox{ donde }t\mbox{ var\'ia en }\R\}$.

\item $A=\{(x,y,z)\in\R^3:\ x^2+y^2=1\}$.

\item $A=\{(x,y,z)\in\R^3:\ 3x^2+y^2=2,\ -1\leq z \leq 2\}$.

\item $A=\{(x,y,z)\in\R^3:\ x^2-y^2=1,\ z=1\}$.

\item $A=\{(x,y,z)\in\R^3:\ x^2+y^2+(z-2)^2=10\}$.

\item $A=\{(x,y,z)\in\R^3:\ x^2+y^2+z^2=1, (x-1)^2+y^2=1\}$.

\item $A=\{(x,y,z)\in\R^3:\ 2x^2+y^2+z^2=2, x+y=1\}$.

\item $A=\{(x,y,z)\in\R^3:\ x^2+y^2-4z^2=1\}$.

\item $A=\{(x,y,z)\in\R^3:\ x=10 \cos(\theta), y= 10\sen(\theta),\ \theta\in[0,2\pi]\}$.

\item $A=\{(x,y,z)\in\R^3:\ x=5\sen(\phi)\cos(\theta),\ y=5\sen(\phi)\sin(\theta),\ z=5\cos(\phi), \phi\in[-\pi,\pi] \mbox{ y }\theta\in [0,\pi]\}$.

\end{enumerate}

\item Dados los vectores $u=(1,1,1),\ v=(2,1,0)$ de $\R^3$ hallar

$\proba{\parallel u + v \parallel} \proba{\text{dist}(u,v)}
\proba{\left<u,v\right>} \proba{\mbox{ el \'angulo entre } u \mbox{ y } v}$\\
$\proba{\mbox{ la recta } r \mbox{ paralela a }u\mbox{ por }(0,0,1)}  \proba{\mbox{ el plano } \Pi \mbox{ perpendicular a }v\mbox{ por }(0,0,0)}$\\
$\proba{r\cap \Pi}.$

\cuento

\vspace{0,4cm}

\item Muestre que en general dado un vector $v$ del espacio $\R^3$, el conjunto $K_v$ de puntos cuya proyecci\'on sobre $v$ es cero es un plano que pasa por el origen.
Encuentre dicho plano para $v=(1,-1,0)$. Muestre que dados dos vectores $v$ y $v'$, dichos planos coinciden si y solo si $v$ y $v'$ son colineales. 

\item Encuentre una fórmula para medir la distancia entre un el hiperplano $a_1x_1+\dots+a_nx_n=b$ y el vector $v=(v_1,\dots, v_n)$ en $\mathbb{R}^n$.

\item (\emph{Las $p$-normas en $\RR^n$} ) Para un real $p\geq 1$ defina la función 
$\|\bullet\|_p: \RR^n\rightarrow \RR$ dada por \[\|x\|_p:=\left(\sum_{i=1}^n |x_i|^p\right)^{\frac{1}{p}}.\]
\begin{enumerate}
\item Demuestre que $\|\bullet\|_p$ es una norma en $\RR^n$ para todo valor de $p\geq 1$.
\item Para $n=2$ y $p=1,2,3,20$ dibuje la bola unitaria de la norma $p$, es decir el conjunto
\[\{(x,y)\in \RR^2: \|(x,y)\|_p\leq 1\}\]

\item Demuestre que $\lim_{p\rightarrow \infty}\|x\|_p=\max_{i=1,2,\dots,n}|x_i|$.

\item Para $n=3$ y $p=1,\infty$ dibuje la bola unitaria de la norma $p$, es decir el conjunto
\[\{(x,y,z)\in \RR^3: \|(x,y)\|_p\leq 1\}\]

\item Demuestre que para la norma euclídea se cumple que $b(x,y):=(\|x+y\|_2-\|x\|_2-\|y\|_2)/2$ es una función bilineal simétrica. 

\item Demuestre que  $b_p(x,y):=(\|x+y\|_p-\|x\|_p-\|y\|_p)/2$ es bilineal simétrica si y solo si $p=2$.
\end{enumerate}


\item  Para una matrix $A \in \mathbb{R}^{m \times n}$ la norma de Frobenius se define como 
\[
\|A\|_F = \sqrt{\sum_{i=1}^m \sum_{j=1}^n |a_{ij}|^2} = \sqrt{\operatorname{Tr}(A^\top A)}.
\]
\begin{enumerate}
\item Demuestre que $\|A\|_F$ es una norma en el espacio de matrices.

\item Demuestre que la norma de Frobenius es invariante bajo transformaciones ortogonales. Es decir, pruebe que
\[
\|U A V^\top\|_F = \|A\|_F.
\]
Para cualquier par  \( U \in \mathbb{R}^{m \times m} \), \( V \in \mathbb{R}^{n \times n} \) de matrices ortogonales, es decir que satisfacen las ecuaciones \( U^\top U = I_m \) y \( V^\top V = I_n \).

\item Calcule la norma de Frobenius de una matriz de permutación.

\end{enumerate}

 
\end{enumerate}




\end{document}
