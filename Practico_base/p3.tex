\documentclass[11pt]{article}

\usepackage[spanish]{babel}

\usepackage{amsfonts, amsthm, amssymb, amsmath,multicol}

\usepackage{graphics,graphicx}

\headsep17mm
\topmargin-1cm
\hoffset -1.5cm \voffset -1cm \textwidth 17cm \oddsidemargin
1.5cm \evensidemargin 1.5cm \textheight 22.5cm

\newcommand{\coef}[2]{\left( \begin{array}{c} #1 \\ #2 \end{array}\right)}

\newcounter{cuent}
%\newcommand{\proba}[1]{\stepcounter{cuent}{\alph{cuent})\quad}
%\displaystyle#1\qquad}
\newcommand{\proba}[1]{\stepcounter{cuent}{\alph{cuent})\ }
\displaystyle#1\hfill}
\newcommand{\cuento}{\setcounter{cuent}{0}}
\newcommand{\R}{\mathbb{R}}

\begin{document}

\noindent {\bf Universidad de la Rep\'{u}blica} \hfill {\bf C\'{a}lculo 2} \\
{\bf Facultad de Ciencias} \hfill {\bf Segundo semestre 2024} \\
{\bf Centro de Matem\'{a}tica} 

\vspace{1cm}

\begin{center}
{\bf Pr\'{a}ctico 3 }
\end{center}

\vspace{0,1cm}

\begin{enumerate}

\item \begin{enumerate}


       \item Sean $f,g:D\subseteq\R^n\rightarrow\R$, y $a\in
         D'$. Probar que si $\lim_{x\rightarrow a}f(x)=0$ y $g$ es una 
         funci\'on acotada en alguna bola reducida centrada en $a$,
         entonces $\lim_{x\to a}f(x)g(x)=0$.  
       \item Calcular: $\displaystyle{\lim_{(x,y)\to (0,0)} x\sen
           \frac{1}{x^2 +y^2 }}$, 
         $\displaystyle{\lim_{(x,y)\to (0,0)}\frac{xy^{2}}{x^2
             +y^2}}$, y 
         $\displaystyle{\lim_{(x,y)\to (0,0)} % \qquad
         \frac{xy^{3}}{x^2 +y^4}}.$
      \end{enumerate}
       
\item\label{bonus4-1} Calcular los siguientes l\'\i mites:
      \[ a\textrm{)}\lim_{(x,y)\to (1,2)}\frac {x^2 +xy+1}{x^2 -x-y} \quad 
         c\textrm{)}\lim_{(x,y)\to (0,0)}xy\log|y| \quad
         e\textrm{)}\lim_{(x,y)\to (1,1)}\frac{x^2 +xy-2y^2 }{x^2-y^2}\] 
      \[ b\textrm{)}\lim_{(x,y)\to (0,0)}\frac{x^3 +y^3 }{x^2 +y^2} \quad 
         d\textrm{)}\lim_{(x,y)\to (0,0)}\frac{e^{x-y}-1 }{x^2 -y^2} \quad
         f\textrm{)}\lim_{(x,y)\to (0,0)}\frac{\log(1+x^2 +y^2) }{x^2 +y^2 +
           x^{3}y}\] 
\item Probar que en los siguientes casos no existe el
      $\lim_{(x,y)\rightarrow (0,0)} f(x,y)$: 
      \[ f(x,y)=\frac{x^2 -y^2}{x^2 +y^2} \qquad
      f(x,y)=\frac{2x^{3}y}{(x^{2}+y^{2})^{2}}  \qquad f(x,y)=\left\{  
      \begin{array}{ll} \frac{xy}{x+y} & \textrm{ si }x+y\neq 0 \\ 0
        &\textrm{ si }
        x+y=0  \end{array} \right. \]

\item \begin{enumerate}

\item Probar que si $\lim_{(x,y)\rightarrow (a,b)}f(x,y)=L$ y
      existen los l\'\i mites unidimensionales   
      $\lim_{x\rightarrow a}f(x,y)$ (para cada $y$) y  \ $\lim_{y\rightarrow
      b}f(x,y)$ (para cada $x$), entonces tambi\'en existen  los l\'\i mites iterados  
      $\lim_{x\to a}\ \left( \lim_{y\to b}f(x,y)\right)$ y $ 
      \lim_{y\to b}\left( \lim_{x\to a}f(x,y) \right)$, y son iguales a 
      $L$. 
      \item Verificar que los l\'\i mites iterados de
      $f(x,y)=(x-y)/(x+y)$ existen en $(0,0)$ pero son distintos.
      \item Verificar que los l\'\i mites iterados de
      $f(x,y)=(x^{2}y^2)/(x^{2}y^{2}+(x-y)^2)$ existen y son iguales en
      $(0,0)$ pero no existe el $\lim_{(x,y)\rightarrow
      (0,0)}f(x,y)$.  
      \item Se considera $f:\R^2\to\R$ definida por:
      $ f(x,y)=\begin{cases} x\sen(1/y) &\textrm{ si } y\neq 0\\ 0 
      &\textrm{ si }y=0.\end{cases}.$\\
      Mostrar que el $\lim_{(x,y)\rightarrow (0,0)}f(x,y)=0$, pero un
      l\'\i mite iterado no existe. ?`Esto contradice la parte
      \textit{a})?  
      \end{enumerate}
              
\item Dada una funci\'on $f:\R^2\rightarrow\R$, se considera el cambio de
      variable (a {\em coordenadas polares}) $x=r\cos\theta$, $y=r\sen\theta$, con
      $\theta\in[0,2\pi)$, $r\geq 0$, y obtenemos $f(x,y)=g(r,\theta)$,
      $g:[0,\infty)\times[0,2\pi)\to\R$.  
      \begin{enumerate}
      \item Aplicar dicho cambio de variable para calcular:
      
      $ \lim_{(x,y)\rightarrow (0,0)}\frac{x^{2}y}{x^2 +y^2 }$ \quad y \quad 
        $\lim_{(x,y)\rightarrow (0,0)}\frac{xy}{\sqrt{x^2 +y^2}
        }$.
      \item ?`Es cierto que si $\lim_{r\to0}g(r,\theta)=0$ para todo $\theta\in [0,2\pi)$, entonces $\lim_{(x,y)\rightarrow (0,0)}f(x,y)=0$? Seg\'un corresponda, demostrar o dar un contraejemplo.
      \end{enumerate}

\item \label{bonus4-2}Sean $f:A\subseteq\R^n \longrightarrow \R^m$,
      $a$ punto de acumulaci\'on de $A$, $g: B\to\R^p$, $b$ de
      acumulaci\'on de $B$ con   
      $f(A)\subset B\subset\R^m$. Supongamos que $ \lim_{x\to
      a}f(x)=b$, $\lim_{y\to b}g(y)=c.$ 
      ?`Qu\'e  se puede decir acerca de $\lim_{x\to
      a}g\big(f(x)\big)$? Discutir el asunto y justificar la
      respuesta.         
       
      
\item En los siguientes casos, hallar el conjunto de los puntos en los que $f$ es continua:

a) $f(x,y)=\left\{ 
\begin{array}{ccc}
\frac{x^{2}y^{3}}{x^{2}+y^{2}} & \text{si} & (x,y)\neq \left( 0,0\right)  \\ 
1 & \text{si} & \left( x,y\right) =\left( 0,0\right) 
\end{array}%
\right. $ b) $f(x,y)=\left\{ 
\begin{array}{ccc}
\frac{x^{2}y^{3}}{x^{2}+y^{2}} & \text{si} & (x,y)\neq \left( 0,0\right)  \\ 
0 & \text{si} & \left( x,y\right) =\left( 0,0\right) 
\end{array}%
\right. $ 

c) $f(x,y)=\left\{ 
\begin{array}{ccc}
\frac{xy}{x^{2}+xy+y^{2}} & \text{si} & (x,y)\neq \left( 0,0\right)  \\ 
0 & \text{si} & \left( x,y\right) =\left( 0,0\right) 
\end{array}%
\right. $ d) $f(x,y)=\left\{ 
\begin{array}{ccc}
\frac{e^{x^{2}y}-1}{\sqrt{x^{2}+y^{2}}} & \text{si} & (x,y)\neq \left(
0,0\right)  \\ 
0 & \text{si} & \left( x,y\right) =\left( 0,0\right) 
\end{array}%
\right. $.

\item  Estudiar la continuidad uniforme de las funciones siguientes en
       los dominios dados: 
       \begin{enumerate}
       
       
       
       \item $f(x)=\cos\big(x^2\big)$ en los intervalos $[0,2\pi]$
                y $\R$.   
          \item $f(x,y)=1/{(x+y^2)}$ en el cuadrado $(0,1)\times
            (0,1)$. \end{enumerate}


\item \begin{enumerate}

 
     \item Probar que la composici\'on de funciones uniformemente 
           continuas tambi\'en lo es.
     \item Probar que si $f:D\subseteq\R^n\to\R^m$ es
           uniformemente 
           continua, entonces $f$ admite una extensi\'on continua
           (\'unica) a la clausura de $D$. 
           
           {\bf Sugerencia:} recordar que $f$ transforma
           sucesiones de Cauchy en sucesiones de Cauchy.
      \end{enumerate}
        

        
\end{enumerate}



\end{document}
