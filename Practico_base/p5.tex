\documentclass[11pt]{article}

\usepackage[spanish]{babel}

\usepackage{amsfonts, amsthm, amssymb, amsmath,multicol}

\usepackage{graphics,graphicx}

\headsep17mm
\topmargin-1cm
\hoffset -1.5cm \voffset -1cm \textwidth 17cm \oddsidemargin
1.5cm \evensidemargin 1.5cm \textheight 22.5cm

\newcommand{\coef}[2]{\left( \begin{array}{c} #1 \\ #2 \end{array}\right)}

\newcounter{cuent}
%\newcommand{\proba}[1]{\stepcounter{cuent}{\alph{cuent})\quad}
%\displaystyle#1\qquad}
\newcommand{\proba}[1]{\stepcounter{cuent}{\alph{cuent})\ }
\displaystyle#1\hfill}
\newcommand{\cuento}{\setcounter{cuent}{0}}
\newcommand{\R}{\mathbb{R}}

\begin{document}

\noindent {\bf Universidad de la Rep\'{u}blica} \hfill {\bf C\'{a}lculo 2} \\
{\bf Facultad de Ciencias} \hfill {\bf Segundo semestre 2024} \\
{\bf Centro de Matem\'{a}tica} 

\vspace{1cm}

\begin{center}
{\bf Pr\'{a}ctico 5 }
\end{center}

\vspace{0,1cm}

\begin{enumerate}

\item Hallar la aproximaci\'{o}n lineal de $f(x,y)=\sqrt{20-x^{2}-7y^{2}}$ en $%
\left( 2,1\right) $ y usarla para calcular de manera aproximada $%
f(1.95,1.08).$

\item Calcular el campo gradiente de las siguientes funciones. Cuando sea posible, representar gr\'aficamente.

a) $f(x,y)=x-y$, \ \ b) $f(x,y)=\ln(x^2+y^2)$, \ \ c) $f(x,y)=x\sen(y)+1$, \ \ d) $f(x,y)=\sen(x^2-y^2)$

e) $f(x,y,z)=2x+y-3z$, \ \ f) $f(x,y,z)=x^2-y^2+z^2$, \ \ g) $f(x,y,z)=e^{x+y}-z$ \\
h) $f(x,y,z)= x\cos(y)$, \ \ i) $f(x_1,\dots,x_n)=x_1+\dots+x_n$ \ \ j) $f(x_1,\dots,x_n)=x_1^2+\dots+x_n^2$.  

\item Hallar el espacio tangente, dando su direcci\'on perpendicular, de los siguientes conjuntos de nivel en los puntos indicados:

a) $x-y=1$ en $(1,0)$, \ \ b) $2x^2+y^2=3$ en $(1,1)$, \ \ c) $\ln(x^2+y^2)=0$ en $(\frac{1}{\sqrt{2}},-\frac{1}{\sqrt{2}})$,

d) $x\sen(y)+1=3$ en $(2,\frac{\pi}{2})$, \ \ e) $1=\sen(x^2-y^2)$ en $(\sqrt{\frac{\pi}{2}},0)$, \ \ f) $x^2+y^2+z^2=6$ en $(1,1,2)$, \ \ g) $x^2-3y^2+z^2=2$ en $(1,1,2)$ \ \ h) $e^{xy}\cos(z)=0$ en $(0,1,\frac{\pi}{2})$,\\
i) $x_1^2+\dots+x_n^2=n$ en $(1,\dots,1)$, \ \ j) $x_1x_4-x_2x_3x_1+x_3x_2=1$ en $(1,1,1,1)$. 

       
\item\label{bonus6-2} Si $(x_0,y_0,z_0)$ es un punto de la superficie
  $z=xy$, las dos 
  rectas $z=y_0x,\,y=y_0$ y $z=x_0y,\,x=x_0$, se cortan en
  $(x_0,y_0,z_0)$. Comprobar que el plano tangente a esta superficie
  en el punto $(x_0,y_0,z_0)$ contiene a esas dos rectas. 

\item Hallar la ecuaci\'on de la recta que es tangente en el punto
  (1,1,1) a las dos superficies $x^2+y^2+2z^2=4$ y $z=e^{x-y}$. 

\item

\begin{enumerate}\item Hallar un vector $V(x,y,z)$ normal a la superficie 
  $z=\sqrt{x^2+y^2}+(x^2+y^2)^{3/2}$ en un punto cualquiera
  $(x,y,z)\neq (0,0,0)$ de la superficie. 
     \item Hallar el coseno del \'angulo $\theta$ formado por el
       vector $V(x,y,z)$ y el eje $z$, y determinar el l\'\i mite de
       $\cos\theta$ cuando $(x,y,z)\to (0,0,0)$. \end{enumerate}

\item Calcular la derivada direccional de $f(x,y,z)=x^2+y^2-z^2$ en
  $(3,4,5)$ a lo largo de la curva de intersecci\'on de las dos
  superficies $2x^2+2y^2-z^2=25$ y $x^2+y^2=z^2$. 

\item Si $f:\R^n\to\R$ es de clase $C^1$ y $f(0)=0$, probar que existen
  $g_i:\R^n\to\R$ continuas tales que $f(x)=\sum_{i=1}^nx_ig_i(x)$. 
  
 Sugerencia: si $h_x(t)=f(tx)$, entonces $f(x)=\int_0^1h'_x(t)dt$.
 
\item En cada caso, calcular la matriz Jacobiana de las funciones $f$ y $g$ en cada punto de sus respectivos dominios. Hallar la funci\'on compuesta $h=f\circ g$ y su matriz Jacobiana en los puntos indicados:

\begin{enumerate}
\item $f(x,y)=(e^{x+2y},\sen(y+2x))$, y
  $g(u,v,w)=(u+2v^2+3w^3, 2v-u^2)$. Hallar $Jh(1,-1,1)$.
  
  \item $f(x,y,z)=(x^2+y+z,2x+y+z^2)$, y 
  $g(u,v,w)=(uv^2w^2,w^2\sen v,u^2e^v)$. Hallar $Jh(u,0,w)$.
    
\end{enumerate}

\item Determinar la soluci\'on de la ecuaci\'on en derivadas parciales
  $4f_x+3f_y=0$ que satisfaga la condici\'on $f(x,0)=\sen x$ para todo
  $x$.   
\end{enumerate}



\end{document}
