\documentclass[11pt]{article}

\usepackage[spanish]{babel}

\usepackage{amsfonts, amsthm, amssymb, amsmath,multicol}

\usepackage{graphics,graphicx}

\headsep17mm
\topmargin-1cm
\hoffset -1.5cm \voffset -1cm \textwidth 17cm \oddsidemargin
1.5cm \evensidemargin 1.5cm \textheight 22.5cm

\newcommand{\coef}[2]{\left( \begin{array}{c} #1 \\ #2 \end{array}\right)}

\newcounter{cuent}
%\newcommand{\proba}[1]{\stepcounter{cuent}{\alph{cuent})\quad}
%\displaystyle#1\qquad}
\newcommand{\proba}[1]{\stepcounter{cuent}{\alph{cuent})\ }
\displaystyle#1\hfill}
\newcommand{\cuento}{\setcounter{cuent}{0}}
\newcommand{\R}{\mathbb{R}}
\newcommand{\be}{\begin{enumerate}}
\newcommand{\ee}{\end{enumerate}}

\begin{document}

\noindent {\bf Universidad de la Rep\'{u}blica} \hfill {\bf C\'{a}lculo 2} \\
{\bf Facultad de Ciencias} \hfill {\bf Segundo semestre 2024} \\
{\bf Centro de Matem\'{a}tica} 

\vspace{1cm}

\begin{center}
{\bf Pr\'{a}ctico 6 }
\end{center}

\vspace{0,1cm}

\begin{enumerate}



\item Para las siguientes funciones calcular los polinomios de Taylor de órdenes 1 y 2 en el punto indicado. %Bosquejar gráfico del diferencial primero y diferencial segundo.

a) $f(x,y)=\sen(x)\sen(y)$ en $(0,0)$ \ \ \  b) $f(x,y)=\sqrt{5x+2y}$ en $(1,1)$ \ \ \  c) $f(x,y)=e^{x^2+y(y+1)}$ en $(0,0)$ 

\item Para las siguientes funciones calcular los polinomios de Taylor de órden $n$ (arbitrario) en el punto indicado.

a) $f(x,y,z)=xyz$ en $(1,-1,0)$ \ \ \ b) $f(x,y)=e^{x+y}$ en
  $(0,0)$ y $(1,0)$.

\item Calcular los siguientes l\'imites:

a) $\underset{\left( x,y\right) \rightarrow \left( 0,0\right) }{\lim}\frac{xy-\sen(x)\sen(y)}{x^2+y^2}$\ \ 
b) $\underset{\left( x,y\right) \rightarrow \left( 0,0\right) }{\lim}\frac{e^{x^2+y(y+1)}-(1 + y)}{x^2+y^2}$.



\item Hallar y clasificar los puntos cr\'\i ticos. En caso de que existan, hallar el m\'aximo y el m\'\i nimo
  absolutos:
  \be\item $f(x,y) = x^4 + y^2 + y^4.$ 
     \item $f(x,y) = 1 - y^2 -x^4.$ 
     \item $f(x,y) = x^4 + y^4 -2x^2 +4xy -2y^2.$ 
     \item $f(x,y) = (ax^2 + by^2)e^{-(x^2 + y^2)}.$ 
     \item $f(x,y) = (x^2 + y^2 -2x+1)/(x^2+y^2+2x-2y+3).$ 
     \item $f(x,y,z)=x^2+3y^2+2z^2-2xy+3xz$.
\item  $f(x,y) = (3-x)(3-y)(x+y-3).$
\item $f(x,y) = xy(1-x^2 - y^2)$ en $[0,1]^2$. 

\item $f:\R^n\times \R^m \to \R$ definida mediante  
  $f(x,y) = \langle x,x\rangle - \langle y,y\rangle$, con $x\in \R^n$, $y \in \R^m$.
  
  
  \ee
\item Hallar y clasificar los puntos cr\'\i ticos de las siguientes 
  funciones:
  \be \item $f(x,y) = x^2 + (y-1)^2.$ 
      \item $f(x,y) = 1 - y^2 +x^2.$ 
      \item $f(x,y) = (x-y-1)^2.$ 
      \item $f(x,y) = x^3 - 3xy^2 + y^3.$ 
      \item $f(x,y) = x^3 + y^3 -3xy.$ 
      \item $f(x,y) = \sen(x)\sen(y)\sen(x+y)$, solamente para los puntos que est\'en en $[0,2\pi]\times
        [0,2\pi]$. 
      \item $f(x,y,z)=x^2+y^2-z^2$.
      \item $f(x,y,z) = x^4 + y^4 + z^4 - 4xyz$
      \item $f(x,y)=(x^2 + y^2)e^{x^2-y^2}$, en $D=\{ (x,y) \in \R^2 : x^2+y^2 < 1 \}$.
\item $f(x,y)=x^4+2x^2y-2y^2-3x^2+1$, en $D=\{(x,y) \in \R^2 : 0\leq x \leq 2, 0 \leq y \leq x^2 \}$.
  \ee
  


\item  Hallar $a,b\in \R$  para que el valor de la integral
$\int_{-1}^{1}(x^2-ax-b)^2dx$ sea m\'\i nimo.

\item Dada $\varphi:(a,b)\to\R$ derivable, definimos
  $f:(a,b)\times(a,b)\to\R$ poniendo
  $f(x,y)=\int_x^y\varphi(t)dt$. 
\begin{enumerate}
\item Hallar los puntos cr\'\i ticos de
  $f$, y determinar la condici\'on para que sean no degenerados.
  \item Para $\varphi(t)=3t^2-1$, hallar y clasificar los puntos cr\'iticos.
\end{enumerate}  
   


\item Sean $I$ un intervalo abierto en $\R$ y $f:I\times[a,b]\subset 
  \R^2 \to \R$ una funci\'on continua tal que existe y es continua 
  $\frac{\partial f}{\partial x}:I\times [a,b]\to \R$.
  \be
   \item Se define $G: (a,b)^2\times I \subset \R^3 \to \R$
     mediante $G(u,v,x) = \int_u^vf(x,y)dy$. Calcular $\frac{\partial
       G}{\partial u}$, $\frac{\partial G}{\partial v}$ y
     $\frac{\partial G}{\partial x}$.
   %\item Sean $h_1,h_2:I\to \R$ funciones de clase $C^1$ tales que
    % $\im(h_1) \subset [a,b]$ e $\im(h_2) \subset[a,b]$. Definimos $F:I
     %\to \R$ mediante $F(x) = \int_{h_1(x)}^{h_2(x)}f(x,y)dy$. 
  %\ee
 % Probar que: $F'(x) = f(x,h_2(x))h_2'(x) - f(x,h_1(x))h_1'(x) + 
  %\int_{h_1(x)}^{h_2(x)}\frac{\partial f}{\partial x} (x,y)dy$. 

\item\label{bonus7-1} Sea $f:\R \to \R$ definida por $f(x) =
  \int_0^x\sen(xy)dy$. Sin  
  calcular la integral, probar que \[f'(x) = \sen(x^2) + \int_0^x 
  y\cos(xy)dy \]
  
 \ee 
 
 \item \textit{Regresi\'on lineal.}  Dados $(x_1,y_1),\dots
  ,(x_n,y_n)\in\R^2$, hallar una funci\'on af\'\i n $f:\R\to\R$, es
  decir, tal que $f(x) = ax + b,\ \forall x\in\R$, que minimice el
  error cuadr\'atico $E(a,b)$, dado por  
  $E(x,y) = \sum_{i = 1}^n (f(x_i) - y_i)^2$.


\end{enumerate}

{\Large \bf Ejercicios optativos}

\begin{enumerate}


  \item Sea $f:\R^n\to\R$ continua, con $n\geq 2$. Si para
  alg\'un $c\in\R$ el conjunto de nivel $f^{-1}(c)$ es compacto, mostrar que
  entonces $f$ posee al menos un extremo absoluto (m\'aximo o m\'inimo). Notar que para $n=1$ esto no es cierto.
  
  
  \item Si $\mathbf{r_1}$ y $\mathbf{r_2}$ son las distancias desde un
  punto $(x,y)$ de una elipse a sus focos, demostrar que la ecuaci\'on
  $\mathbf{r}_1+\mathbf{r}_2=$constante (que satisfacen esas
  distancias) implica la relaci\'on
  $T\cdot\nabla(\mathbf{r}_1+\mathbf{r}_2)=0$, siendo $T$ el vector
  unitario tangente a la elipse en $(x,y)$. Interpretar geom\'etricamente este
  resultado, y con ello demostrar que la tangente forma \'angulos
  iguales con las rectas que unen $(x,y)$ a los focos. 
\end{enumerate}
\end{document}
