\documentclass[11pt]{article}

\usepackage[spanish]{babel}

\usepackage{amsfonts, amsthm, amssymb, amsmath,multicol}

\usepackage{graphics,graphicx}

\headsep17mm
\topmargin-1cm
\hoffset -1.5cm \voffset -1cm \textwidth 17cm \oddsidemargin
1.5cm \evensidemargin 1.5cm \textheight 22.5cm

\newcommand{\coef}[2]{\left( \begin{array}{c} #1 \\ #2 \end{array}\right)}

\newcommand{\be}{\begin{enumerate}}
\newcommand{\ee}{\end{enumerate}}

\newcounter{cuent}
%\newcommand{\proba}[1]{\stepcounter{cuent}{\alph{cuent})\quad}
%\displaystyle#1\qquad}
\newcommand{\proba}[1]{\stepcounter{cuent}{\alph{cuent})\ }
\displaystyle#1\hfill}
\newcommand{\cuento}{\setcounter{cuent}{0}}
\newcommand{\R}{\mathbb{R}}

\begin{document}

\noindent {\bf Universidad de la Rep\'{u}blica} \hfill {\bf C\'{a}lculo 2} \\
{\bf Facultad de Ciencias} \hfill {\bf Segundo semestre 2024} \\
{\bf Centro de Matem\'{a}tica} 

\vspace{1cm}

\begin{center}
{\bf Pr\'{a}ctico 9 }
\end{center}

\vspace{0,1cm}

\begin{enumerate}

\item Calcular las siguientes integrales en los bloques dados.
      \be
        \item $\int_Ixy(x+y)$, donde $I=[0,1]\times[0,1]$.
        \item $\int_I(x^3+3x^2y+y^3)$, donde $I=[0,1]\times[0,1]$.
        \item $\int_I(\sqrt{y}+x-3xy^2)$, donde $I=[0,1]\times[1,3]$.
        \item $\int_I\sen(x)\,\sen(y)$, donde $I=[0,\pi]\times[0,\pi]$. 
        \item $\int_Iy^{-3}e^{tx/y}$, $I=[0,t]\times[1,t]$.
        \item $\int_If(x,y)$, donde $I=[-1,1]\times[-1,1]$,
          $f(x,y)=\begin{cases}x^2+y^2&\textrm{ si }x^2+y^2\leq 1\\
            0&\textrm{ en otro caso.}\end{cases}$.               
        \item $\int_Ize^{y-zx}$, donde $I=[0,1]\times[0,1]\times
          [0,1]$. 
        \item\label{bonus11a} $\int_I(y\cos x\,\sen z+ x^2ye^{xyz})$,
          donde $I=[2\pi,5\pi/2]\times[0,1]\times [0,\pi]$. 

      \ee
\item Las integrales iteradas que siguen corresponden a integrales
  de ciertas funciones $f$ definidas sobre ciertos dominios. Dibujar
  esos dominios y expresar las integrales como integrales iteradas en
  el orden inverso de integraci\'{o}n. 
\[ 
\int_{0}^{1}\,dy \int_{0}^{y}f(x,y)\,dx \ \ \ \ \ \ \
\int_{1}^{4}\,dx \int_{\sqrt{x}}^{2}f(x,y)\,dy \ \ \ \ \ \ \
\int_{0}^{2}\,dy \int_{y^{2}}^{2y}f(x,y)\,dx 
\]
\[ 
\int_{1}^{2}\,dx \int_{2-x}^{\sqrt{2x-x^{2}}}f(x,y)\,dy \ \ \ \ \ \ \ \ 
\int_{1}^{e}\,dx \int_{0}^{\log(x)}f(x,y)\,dy   
\]
\item Calcular $\iint_{D}f(x,y)\,dx\,dy$ en cada uno de los siguientes
  casos: 
     \be
      \item $f(x,y)=2x-y $ \ y \ $D=\{ (x,y) \in \R^{2}\colon 1\leq x 
            \leq 4 , 0 \leq y \leq \sqrt{x} \}$.
      \item $f(x,y)=\sqrt{4x^{2}-y^{2}} $ \ y \ $D=\{ (x,y) \in 
            \R^{2}\colon 0\leq x \leq 1 ,      0 \leq y \leq \ x \}$. 
      \item $f(x,y)=xy^{2} $ \ y \ $D=\{ (x,y) \in \R^{2}\colon 0\leq
            y \leq 1 , y \leq x \leq  y+1 \}$.
      \item $f(x,y)=x^{2}-y^{2} $ \ y \ $D=\{ (x,y) \in \R^{2}\colon 
            0\leq x \leq \pi ,      0 \leq y \leq \sen x \}$. 
      \item $f(x,y)=xy $ \ y \ $D=\{ (x,y) \in \R^{2}: 
            \frac{x^{2}}{a^{2}}+\frac{y^{2}}{b^{2}}\leq 1,\ x \geq 0,\   y \geq 0 \}$.  
      \item $f(x,y)=(xy)^{2} $ \ y \ $D$ es la regi\'{o}n acotada del 
            primer cuadrante comprendida entre las hip\'erbolas:
            $xy=1$, $xy=2$ y las rectas $y=x$, $y=4x$. 
\end{enumerate}

\item Calcular $\iint_{D} f(x,y)\,dx\,dy$ en cada uno de los
      siguientes casos, haciendo cambios de variable convenientes: 
  \begin{enumerate}
   %\item $f(x,y)=e^{-(x^{2}+y^{2})} $ \ y
   %    \ $D=\{ (x,y) \in \R^{2}\colon x^{2}+y^{2} \leq r^{2} \}$.
   \item $f(x,y)=x+y $ \ y \ $D=\{ (x,y) \in \R^{2}\colon 0\leq x  , \
     0 \leq y \leq \ x , \      x^{2}+y^{2} \leq 1  \}$. 
   \item $f(x,y)=x$, y $D$ es el
      paralelogramo de v\'ertices $(-2/3,-1/3)$, $(2/3,1/3)$,
      $(4/3,-1/3)$ y $(0,-1)$.  
   \item $f(x,y)=x^{2}+y^{2} $ \ y \ $D=\{ (x,y) \in \R^{2}\colon
     0\leq y  ,\ x^{2}+y^{2} \geq 1 , \ x^{2}+y^{2}-2x \leq 0  \}$. 
   \item $f(x,y)=x^{2}/(x^{2}+y^{2})$ \ y \ $D$ es el tri\'{a}ngulo de 
         lados $y=x$, $y=-x$, $x=1$ (se sugiere pasar a coordenadas
         polares). 
   \item $f(x,y)=(x-y)^{2}\sen^{2}(x+y) $ \ y \ $D$ es el cuadrado de 
         v\'ertices $(\pi,0)$, $(0,\pi )$, $(2\pi , \pi )$, $(\pi
         ,2\pi )$. 
   \item $f(x,y)=x^{2}/(x^{2}+y^{2}) $ y $D=\{ (x,y) \in \R^{2}\ / 
         \ 0 \leq x \leq 1  , \ x^{2} \leq y \leq 2-x^{2}  \} $. 
         (Se sugiere hacer el cambio de variable $x=\sqrt {v-u}$ ,
         $y=v+u$).
         
         \ee
         
         
    \item En los siguientes casos, calcular el volumen de $D$: 
      \be
       \item $D=\{(x,y,z)\in \R^{3} \colon x^{2}/a^{2}+y^{2}/b^{2}
         \leq z\le 1 \}$. 
       \item $D=\{ (x,y,z)\in \R^{3} \colon  \sqrt{x^{2}+y^{2}} \leq z
         \leq 1 \ \}$.
      % \item $D=\{ (x,y,z)\in \R^{3}\colon  x^{2}+y^{2}+z^{2} \leq r^{2} ,\ x^{2}+y^{2}-rx\geq 0 , \   x^{2}+y^{2}+rx \geq 0 \ \}$. 
       \item $D$ es el conjunto comprendido entre $z=x^{2}$,
         $z=4-x^{2}-y^{2}$. 
       \item $D$ es la intersecci\'on de la bola $x^2+y^2+z^2\le 1$
         con el interior del cilindro $2x^2+y^2-2x=0$. 
       \item $D$ es el s\'olido acotado que limitan $S_1$, $S_2$ y el plano $z=0$,
         donde $S_1$ y $S_2$ son las superficies dadas respectivamente
         por las ecuaciones $2az=x^2+y^2 $ y $x^2+y^2-z^2=a^2$, con
         $a>0$.  
      \ee      
  
 \item Sean $U=\{(u,v)\in \R^{2}\colon u>0\}$ y $h\colon U\rightarrow
      h(U)$ dada por $h(u,v)=(u+v,v-u^{2})$. 
      \be
       \item Probar que $h$ es un cambio de coordenadas (se
             hallar\'{a} expl\'{\i}citamente $h^{-1}$). 
       \item Hallar $J_{h}$ y $\det(J_{h})$ en un punto gen\'erico.  
             Hallar $\det(J_{h^{-1}})$ en $(2,0)$, observando que
             $h(1,1)=(2,0)$. 
       \item Sea $T$ el tri\'{a}ngulo de lados $u=0$, $v=0$, $u+v=2$.  
             Calcular el \'area de $S=h(T)$.
      \ee
\item Calcular $ \iiint_{D}f(x,y,z) \,dx\,dy\,dz$ en los siguientes casos:
\be
  \item $f(x,y,z)=\sqrt{1+(x^2+y^2+z^2)^{3/2}}$, $D=\{(x,y,z)\colon
        x^2+y^2+z^2\leq  R^2\}$.  
  \item $f(x,y,z)=\frac{1}{(x+y+z+1)^{2}},$ $D=\{(x,y,z)\in
        \R^{3}\colon x\geq 0, y\geq 0, z\geq 0, x+y+z \leq 1  \}$. 
  \item $f(x,y,z)=x$ \ y \ $D$ es el dominio acotado limitado por: 
        $z=0$, $y=0$, $y=x$, $x+y=2$, $x+y+z=6$.
  \item\label{bonus11b} $f(x,y,z)=xyz$, \  $D=\{ (x,y,z)\in
        \R^{3}\colon 0 \leq y , \ 
        0 \leq x, \ 0 \leq z , \      x^{2}+y^{2}+z^{2} \leq 1 \}$. 
  \item $f(x,y,z)=x^{2}+y^{2}$ \ y \ $D$ es el dominio acotado
        comprendido entre:      $x^{2}+y^{2}=2x$, $z=0$, $z=2$. 
  \item $f(x,y,z)=(x^{2}+y^{2}+z^{2})^{-1}$ \ y \ $D$ comprendido
        entre: $z=0$, $x=0$, $x=1$, $y=x^{2}$,
        $z=\sqrt{x^{2}+y^{2}}$. 
  \item $f(x,y,z)=\sqrt{x^{2}+y^{2}}$ \ y \ $D$ comprendido entre:
        $z=0$, $z=1$, $z^{2}=x^{2}+y^{2}$. 
  \item $f(x,y,z)=z$ \ y \ $D=\{ (x,y,z)\in \R^{3}\ /\ 
        0 \leq  a \leq x^{2}+y^{2}+z^{2} \leq b \ \}$. 
  \item $f(x,y,z)=\big((x-a)^{2}+(y-b)^{2}+(z-c)^{2}\big)^{-1/2}$  
        \ y \ $D$ la clausura de bola de centro en el origen y radio
        $r$. 
\ee

\item Demostrar la siguiente igualdad: 
      $
      \iint_{D} f(xy) \,dx\,dy=\log(2)\int_{1}^{2}f(u)\ du,  
      $ siendo $D$ la regi\'{o}n del primer cuadrante limitada por
      las hip\'erbolas $xy=1$, $xy=2$ y las rectas $y=x$, $y=4x$. 

\end{enumerate}


\end{document}
