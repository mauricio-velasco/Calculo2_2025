\documentclass[11pt]{article}

\usepackage[spanish]{babel}

\usepackage{amsfonts, amsthm, amssymb, amsmath,multicol}

\usepackage{graphics,graphicx}

\headsep17mm
\topmargin-1cm
\hoffset -1.5cm \voffset -1cm \textwidth 17cm \oddsidemargin
1.5cm \evensidemargin 1.5cm \textheight 22.5cm

\newcommand{\coef}[2]{\left( \begin{array}{c} #1 \\ #2 \end{array}\right)}

\newcommand{\be}{\begin{enumerate}}
\newcommand{\ee}{\end{enumerate}}

\newcounter{cuent}
%\newcommand{\proba}[1]{\stepcounter{cuent}{\alph{cuent})\quad}
%\displaystyle#1\qquad}
\newcommand{\proba}[1]{\stepcounter{cuent}{\alph{cuent})\ }
\displaystyle#1\hfill}
\newcommand{\cuento}{\setcounter{cuent}{0}}
\newcommand{\R}{\mathbb{R}}

\begin{document}

\noindent {\bf Universidad de la Rep\'{u}blica} \hfill {\bf C\'{a}lculo 2} \\
{\bf Facultad de Ciencias} \hfill {\bf Segundo semestre 2024} \\
{\bf Centro de Matem\'{a}tica} 

\vspace{1cm}

\begin{center}
{\bf Pr\'{a}ctico 8 }
\end{center}

\vspace{0,1cm}

\begin{enumerate}

\item En cada caso, encontrar los extremos absolutos de la funci\'on $f$ en el conjunto dado.
 \be
   \item $f(x,y)=ax+by$ en $C:$ $x^2+y^2=1$ (donde $a^2+b^2 \neq 0$). 
   \item $f(x,y,z) = ax+by+cz$ en $S:$ $x^2+y^2+z^2 =1$ (donde $a^2+b^2+c^2
     \neq 0$). 
  % \item $f(x,y)= (x+2y)e^{x+y}$ en $C:$ $x^2+y^2=1$.
   \item $f(x,y)=xy$, en $C=\{(x,y)\in \R^2:\ x+y=1\}$.

\item $f(x,y)=x^2+y^2-xy$, en $C=\{(x,y)\in \R^2:\ x^2+y^2+xy=1\}$.

\item $f(x,y)=xy$, en $D=\{(x,y)\in \R^2:\ 5x^2-6xy+5y^2\leq 4\}$.
\item $f (x, y, z) = xyz$, en $S = \{(x, y, z) : x^2 + y^2 + z^2 = 1\}$. 
  \ee

\item En cada caso, hallar la distancia m\'inima al origen del conjunto dado. Cuando exista, hallar tambi\'en la distancia m\'axima. 
\be
\item $C:5x^2+6xy+5y^2 = 8$ en $\R^2$.
\item  $C:(x-2)^2+(y+3)^2=1$ en $\R^2$.
\item $S:z^2-xy=1$ en $\R^3$.
\item $S:x^2(y+z)+2x(y^2-z^2)+2=0$ en $\R^3$.
\ee

\item En cada caso, hallar la distancia m\'axima y m\'inima de $P\in\R^3$ a la curva $C\subset\R^3$.
\be
\item $P=(1,-1,0)$, $C: x^2+y^2+(z-2)^2=1, x-y+2z=4$ (intersecci\'on de un plano y una esfera).   
\item $P=(0,0,2)$, $C: z^2=x^2+y^2, 3(z-1)^2+(y-1)^2-4=0$ (intersecci\'on de un cono y un cilindro).
\item $P=(1,1,1)$, $C:x^2+y^2-z^2=0, x^2+y^2+(z-1)^2=1$ (intersecci\'on de un cono y una esfera). 
\ee

\item En cada caso, mostrar que el conjunto $C$ es compacto, y hallar los extremos absolutos de la funci\'on $f$ en el conjunto $C$.
\be
\item $C= \{(x,y,z)\in \R^3:3x^2+y^2=12, x+y+z=2\}$, $f(x,y,z)=x+y+2z$.
\item $C= \{(x,y,z)\in \R^3: 4x^2+3y^2-4z=0,y^2-z=0\}$, $f(x,y,z)=4xy+4z-1$.

\item $C=\{(x,y,z)\in \R^3:z-x^2-y^2+2=0, x+y+z-1=0\}$, $f(x,y,z)=xy+xz+yz$. Hallar tambi\'en la distancia m\'axima y m\'inima de $C$ al plano $y+2z=0$.
 
\ee

\item Cu\'al es la cantidad m\'\i nima de cart\'on que se necesita
  para hacer una caja de un litro?
   
\item Demostrar que, entre todos los pol\'\i gonos de $n$ lados
  inscriptos en una circunferencia, el que tiene \'area m\'axima es el
  pol\'\i gono regular. 
  
\item Consideremos la funci\'on $f:\R^n \times \R^n \to \R$ definida
  mediante $f(x,y) = x\cdot y$. Hallar los
  extremos de $f$  sujeta a $\lVert x\rVert^2 + \lVert y \rVert
  ^2 =1$ y deducir de aqu\'\i\ la desigualdad de Cauchy-Schwarz.    

\end{enumerate}


\end{document}
